\section{Verwendung}
 Nach der Installation kann sowohl die Dokumentklasse \verb+HSR+ als auch das \verb+HSRColors+-Package wie die bekannten Klassen und Packages
 verwendet werden.
 Es genügt also folgende Zeilen im Dokument einzufügen:
 \begin{verbatim}
   \documentclass{HSR}
   \usepackage{HSRColors}
 \end{verbatim}
 Wird die Dokumentklasse \verb+HSR+ verwendet erübrigt sich das laden des \verb+HSRColor+-Packages, das Package wird von der \verb+HSR+-Klasse
 automatisch geladen.


\subsection{Dokumentklasse HSR}
Die Dokumentklasse \verb+HSR+ basiert auf der \verb+article+ Klasse und definiert somit das Layout des Dokuments. Der \verb+HSR+ Klasse können also
dieselben Optionen wie der \verb+article+ Klasse übegeben werden. Einige Optionen werden dabei unbeachtet an \verb+article+ weitergereicht, andere
werden dazu verwendet das Layout genauer zu definieren und erst dann weitergereicht. Die Klasse stellt zudem einige nützliche Makros bereit um die
Arbeit weiter zu erleichtern. Mehr dazu ist in den entsprechenden Abschnitten zu finden.

\subsubsection{Definierte Optionen}
Optionen welche das Layout verändern sind der unten stehenden Tabelle zu entnehmen.

\settowidth{\ci}{a4paper}
\setlength{\cii}{\textwidth}
\advance\cii by -\ci
\advance\cii by -1\arrayrulewidth
\advance\cii by -4\tabcolsep

\begin{table}[!hb]
\centering
\rowcolors{2}{HSRLightGray20}{HSRLightGray40}
\begin{tabular}{p{\ci}|p{\cii}}
	\rowcolor{HSRLakeGreen40}
	\textbf{Option} & \textbf{Einfluss}
	\\
	twoside & Wird verwendet um bei beidseitigem Druck die Seitenränder und die Fusszeile anzupassen
	\\
	a4paper & Diese Option wird immer gesetzt und kann im Moment nicht geändert werden.
\end{tabular}
\caption{Optionen mit Einfluss auf das Layout}
\label{tab:DokKlasseHSR:Optionen}
\end{table}

\subsubsection{Definierte Makros}
Um das Arbeiten zu erleichtern und weitere Dokumentoptionen zu setzen wurden zusätzliche Makros definiert.
Tabelle \ref{tab:DokKlasseHSR:Makros} bietet einen kurzen Überblick, auf die einzelnen Makros wird später
noch eingegangen.

\setlength{\ci}{4.5cm}
\settowidth{\cii}{Kein Wert}
\setlength{\ciii}{\textwidth}
\advance\ciii by -\ci
\advance\ciii by -\cii
\advance\ciii by -2\arrayrulewidth
\advance\ciii by -6\tabcolsep

\begin{table}[!ht]
\centering
\rowcolors{2}{HSRLightGray20}{HSRLightGray40}
\begin{tabular}{p{\ci}|p{\cii}|p{\ciii}}
	\rowcolor{HSRLakeGreen40}
	\textbf{Command} 				& \textbf{Default} & \textbf{Beispiel}
	\\
	\verb+\settitle{<Titel>}+ 		& Kein Wert & \verb+\settitle{HSR-Latex-Package}+
	\\
	\verb+\setauthor{<Autor>}+ 		& Kein Wert & \verb+\setauthor{Jürg Rast}+
	\\
	\verb+\setarraystretch{<Wert>}+ & Kein Wert & \verb+\setarraystretch{1.5}+
\end{tabular}
\caption{In der HSR Klasse definierte Makros}
\label{tab:DokKlasseHSR:Makros}
\end{table}



\subsection{Package HSRColors}
